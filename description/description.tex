\documentclass[a4paper]{article}

\usepackage[utf8]{inputenc}

\usepackage{csquotes}

\usepackage{float}

\usepackage[
    backend=biber,
    style=authoryear-icomp,
    natbib=true,
    url=false,
    doi=true,
    eprint=false
]{biblatex}

%\addbibresource{references.bib}

\def\constantMaxdrawdown{42.0}
\def\constantStartdate{1999-11-01 00:00:00}
\def\constantEnddate{2025-01-29 00:00:00}
\def\constantTransactionCommission{0.02}
\def\constantRMean{0.0026}
\def\constantSharpeRatio{0.4906}
\def\constantStd{0.0053}

\usepackage{graphicx}
\usepackage{float}
\usepackage{amsmath}
\usepackage[english]{babel}
\usepackage[hidelinks=true, bookmarks=true]{hyperref}
\usepackage{geometry}

\geometry{
    a4paper,
    left=20mm,
    right=20mm,
    top=20mm,
    bottom=20mm,
}

\def\documenttitle{Description of Bot Tau}
\def\name{Frans Englich}

\title{\documenttitle}
\date{\today}
\author{\name \\
        \href{mailto:fenglich@fastmail.fm}{fenglich@fastmail.fm}}

\hypersetup{
    pdfsubject = {\documenttitle},
    pdftitle = {\documenttitle},
    pdfauthor = {\name},                                                     
    pdfcreator = {\name},                                                    
    pdfproducer = {\name, using \LaTeX}
}

\newcommand{\figureTau}[1]{
    \begin{figure}[H]
        \begin{center}
            \includegraphics{../generated/#1.png}
        \end{center}
        %\caption{}
    \end{figure}
}

\begin{document}

\maketitle

\tableofcontents

\section{Introduction}

This document describes the simulated in-sample performance of Bot Tau's trading
strategy. It does not describe the strategy itself, which is proprietary.

\section{Trading Plan}

The object is to trade assets TODO. We close positions at end of each trading day, because we don't want overnight exposure.
% p. 304

\section{The Dataset}

The dataset stretches from \constantStartdate to \constantEnddate.

\section{Features}

Some form of property, typically derived from the OLHCV. An example is
volatility. The features used are as follows.

\figureTau{feature_BollingerBands}

\figureTau{feature_RSI}

\figureTau{corrmatrix}

\section{Targets}

\section{Model}

\section{Back Test}
\subsection{Drawdown}

Maximum drawdown is \constantMaxdrawdown \%. We consider 20\% an acceptable maximum.

\figureTau{drawdown}

\figureTau{drawdown_dist}

\subsection{Returns}

\begin{table}[H]
\begin{center}
\caption{Statistics of returns.}
    \begin{tabular}{ |l|p{1in}| }
        \hline
        Mean returns            & \constantRMean \%     \\
        \hline
        Standard deviation (SD) & \constantStd          \\
        \hline
        Sharpe Ratio (SR)       & \constantSharpeRatio  \\
        \hline
    \end{tabular}
\end{center}
\end{table}

\figureTau{returns}

The cumulative returns are not compounding, while the annualized returns are.
However, we close the position, meaning compounding isn't relevant.

\figureTau{cumulative_returns}

The transaction cost, $C$, is calculated using the formula, where $t$ is the trade amount:

\begin{equation}
\label{eq:NPV_L}
C = \constantTransactionCommission * t + spread/2
\end{equation}

\figureTau{cumulative_returns_except_trans_costs}

\section{Live Performance}

The plan is to paper trade in a one month incubation period.

TODO compare return dist to back test return using Kolmogorov statistical test. % p. 311.

\subsection{Trading Journal}

No trading have taken place, so nothing here yet.

%\printbibliography
%\appendix

\end{document}
