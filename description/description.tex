\documentclass[a4paper]{article}

\usepackage[utf8]{inputenc}

\usepackage{csquotes}

\usepackage[
    backend=biber,
    style=authoryear-icomp,
    natbib=true,
    url=false,
    doi=true,
    eprint=false
]{biblatex}

\addbibresource{references.bib}

\usepackage{graphicx}
\usepackage{float}
\usepackage{amsmath}
\usepackage[english]{babel}
\usepackage[bookmarks=true]{hyperref}
\usepackage{color}
\usepackage{listings}

\usepackage[
    type={CC},
    modifier={by-nc-sa},
    version={3.0},
]{doclicense}

\usepackage{xcolor}
% New colors defined below.
\definecolor{codegreen}{rgb}{0,0.6,0}
\definecolor{codegray}{rgb}{0.5,0.5,0.5}
\definecolor{codepurple}{rgb}{0.58,0,0.82}
\definecolor{backcolour}{rgb}{0.95,0.95,0.92}

%Code listing style named "mystyle"
\lstdefinestyle{mystyle}{
%  backgroundcolor=\color{backcolour},
  commentstyle=\color{codegreen},
  keywordstyle=\color{magenta},
  numberstyle=\tiny\color{codegray},
  stringstyle=\color{codepurple},
  basicstyle=\ttfamily\footnotesize,
  breakatwhitespace=false,
  breaklines=true,
  keepspaces=true,
  numbers=left,
  numbersep=5pt,
  showspaces=false,
  showstringspaces=false,
  showtabs=false,
  tabsize=2
}

%"mystyle" code listing set
\lstset{style=mystyle}

\def\documenttitle{Description of Bot Tau}

\title{\documenttitle}
\date{\today}
\author{Frans Englich \\
        \href{mailto:fenglich@fastmail.fm}{fenglich@fastmail.fm}}

\hypersetup{
    pdfsubject = {\documenttitle},
    pdftitle = {\documenttitle}
}

\begin{document}

\maketitle

\section{Introduction}

% TODO
% jega_tit_1993 fares well in replicating_anomalies.

% - Actual performance/characteristics
% - Preferred characteristics. Rules, requirements

% * Draw downs. Max 20%
% * Cumulative returns
% * Walk forward
% * Trading costs
% * Sensitivity analysis
% * Transaction costs
% * Thanks
 
% You can visualise/plot:
% A. Various kinds of in-data
% B. Features
% C. Targets (typically a time series, independent variable)
% D. The model
% E. Data on performance
%   E1. With/without costs

% Your strategy/"idea" consists of:
% - Understanding of DGP
% - Statistical/mathematical model

% Your implementation consists of:
% - Data fetch, storage, ETL
% - Model creation
% - Back tests, QA, validation
% - Live running

% Features:
% - Aim your features on the target. What data does that target need? What precedes that target in the data?
% - Regime dependence: Be aware of regime: trending or ranging?
% - Select features based on time: in which time window is the model to
%   act(target horizon)? Calibrate the feature window accordingly.

% Backtesting:
% - Analysis of drawdown
% Lucas:
% - Walk-forward optimization
% - Monte Carlo
% - CPCV
% - Sensitivity tests
% - Stress scenarios
% - Costs, slippage, regime shifts...
% 

% Moment:
% * Is the time series stationary?
% * Is your volatility model correct? Check for volatility clustering. ARCH test.

\begin{equation*}
    \bar{u}_p = \frac{A}{C} + \frac{1}{C} \sqrt{(C\sigma^2_p - 1)(BC - A^2)},
\end{equation*}

\section{Remaining Questions}

\printbibliography

\appendix

\section{Code}

%\lstinputlisting[language=Matlab, caption=Matlab code]{plan.m}

\section{License}

\doclicenseThis

\end{document}
