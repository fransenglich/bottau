\documentclass[a4paper]{article}

\usepackage[utf8]{inputenc}

\usepackage{csquotes}
\usepackage{verbatim}

\usepackage{float}

\usepackage[
    backend=biber,
    style=authoryear-icomp,
    natbib=true,
    url=false,
    doi=true,
    eprint=false
]{biblatex}

%\addbibresource{references.bib}

\def\constantMaxdrawdown{42.0}
\def\constantStartdate{1999-11-01 00:00:00}
\def\constantEnddate{2025-01-29 00:00:00}
\def\constantTransactionCommission{0.02}
\def\constantRMean{0.0026}
\def\constantSharpeRatio{0.4906}
\def\constantStd{0.0053}

\usepackage{graphicx}
\usepackage{float}
\usepackage{amsmath}
\usepackage[english]{babel}
\usepackage[hidelinks=true, bookmarks=true]{hyperref}
\usepackage{geometry}

\geometry{
    a4paper,
    left=20mm,
    right=20mm,
    top=20mm,
    bottom=20mm,
}

\def\documenttitle{Description of Bot Tau}
\def\name{Frans Englich}

\title{\documenttitle}
\date{\today}
\author{\name \\
        \href{mailto:fenglich@fastmail.fm}{fenglich@fastmail.fm}}

\hypersetup{
    pdfsubject = {\documenttitle},
    pdftitle = {\documenttitle},
    pdfauthor = {\name},                                                     
    pdfcreator = {\name},                                                    
    pdfproducer = {\name, using \LaTeX}
}

\newcommand{\figureTau}[1]{
    \begin{figure}[H]
        \begin{center}
            \includegraphics{../generated/#1.png}
        \end{center}
        %\caption{}
    \end{figure}
}

% Fix for \input in tables. https://tex.stackexchange.com/questions/611786/misplaced-noalign-because-input-before-booktabs-rule
\ExplSyntaxOn
\cs_new:Npn \expandableinput #1
  { \use:c { @@input } { \file_full_name:n {#1} } }
\AddToHook{env/tabular/begin}
  { \cs_set_eq:NN \input \expandableinput }
\ExplSyntaxOff

\begin{document}

\maketitle

\tableofcontents

\section{Introduction}

This document describes Bot Tau's trading strategy. Notice that this is one strategy, which is part of a portfolio.

\section{Trading Plan}

\begin{table}[H]
\begin{center}
\caption{Specifics of the trading plan.}
    \begin{tabular}{|l|p{4in}|}
        \hline
        Assets              & Currently undecided \\
        \hline
        Overnight?          & We close positions at end of each trading day, because we don't want overnight exposure. \\
        \hline
        Number of trades per day  &  Currently undecided \\
        \hline
        Performance         &   \begin{itemize}
                                    \item Yearly return $>$ ?
                                    \item Sharp Ratio $>$ ?
                                    \item Calmar Ratio $>$  ?
                                \end{itemize} \\
        \hline
        Over-fitting        & How many times can the strategy be adjusted? How many back tests? \\
        \hline
    \end{tabular}
\end{center}
\end{table}

Risk management conditions:

\begin{itemize}
    \item If we have more than 3 losing trades per day, we stop the algorithm
          for the day.
    \item We stop the algorithm after X \% loss in one month.
    \item We stop the algorithm if the drawdown in live trading becomes  times
          higher than the drawdown in incubation.
\end{itemize}

\section{The Dataset}

The dataset stretches from \constantStartdate \  to \constantEnddate.

\section{Strategy}

\begin{itemize}
    \item Entry signal
    \item Exit signal
    \item Position sizing. Is it static or dynamic in some manner? For instance, percentage of current total capital.
\end{itemize}

\section{Features}

Some form of property, typically derived from the OLHCV. An example is
volatility. The features used are as follows.

\figureTau{feature_BollingerBands}

\figureTau{feature_RSI}

\figureTau{pearsonmatrix}

\figureTau{spearmanmatrix}


\subsection{Multicollinearity}

See:

\begin{itemize}
    \item \url{https://www.geeksforgeeks.org/python/detecting-multicollinearity-with-vif-python/}
    \item \url{https://en.wikipedia.org/wiki/Variance_inflation_factor}
\end{itemize}

Interpretation:

\begin{itemize}
    \item Values near 1 mean predictors are independent.
    \item Values between 1 and 5 shows moderate correlation which is sometime acceptable.
    \item Values above 10 signal problematic multicollinearity requiring action.
\end{itemize}

\begin{table}[H]
\begin{center}
\caption{Variance Inflation Factors (VIF).}
    \begin{tabular}{|l|l| }
        \hline
        Feature & VIF \\
        \hline

        pct\_close\_futur & 1.08 \\
var & 7.76 \\
parkinsons\_var & 7.6 \\


        \hline

    \end{tabular}
\end{center}
\end{table}

\section{Targets}

\section{Model}

\verbatiminput{../generated/ols_conditions.txt}

\section{Back Test}
\subsection{Drawdown}

Maximum drawdown is \constantMaxdrawdown \%. We consider 20\% an acceptable maximum.

\figureTau{drawdown}

\figureTau{drawdown_dist}

\subsection{Returns}

This is the returns of our trading strategy.

\begin{table}[H]
\begin{center}
\caption{Statistics of returns.}
    \begin{tabular}{ |l|p{1in}| }
        \hline
        Mean returns            & \constantRMean \%     \\
        \hline
        Standard deviation (SD) & \constantStd          \\
        \hline
        Sharpe Ratio (SR)       & \constantSharpeRatio  \\
        \hline
        Calmar Ratio (CR)       & \constantCalmarRatio  \\
        \hline
    \end{tabular}
\end{center}
\end{table}

\figureTau{returns}

The cumulative returns are not compounding, while the annualized returns are.
However, we close the position, meaning compounding isn't relevant.

\figureTau{cumulative_returns}

The transaction cost, $C$, is calculated using the formula, where $t$ is the trade amount:

\begin{equation}
\label{eq:NPV_L}
C = \constantTransactionCommission * t + spread/2
\end{equation}

\figureTau{cumulative_returns_except_trans_costs}

\subsection{Further Robustness Tests}

\section{Live Performance}

The plan is to paper trade in a one-month incubation period.

TODO compare return dist to back test return using Kolmogorov statistical test. % p. 311.

\subsection{Performance Report}

(Copy Discord report.)

\subsection{Trading Journal}

No trading have taken place, so nothing here yet.

%\printbibliography
%\appendix

\end{document}
